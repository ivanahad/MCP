\documentclass[a4paper, 12pt]{report}

\usepackage[french]{babel} 
\usepackage[utf8]{inputenc}
\usepackage[T1]{fontenc} 
\usepackage{amsmath}
\usepackage{amssymb}
\usepackage{listings}  

\author{David \bsc{Haven} \and Eddy \bsc{Ndizera} \and Ivan \bsc{Ahad} \and Julien \bsc{Sterbelle}}

\title{Rapport Projet MCP}

\date{\today}

\begin{document}

\maketitle

\section*{Introduction}

Dans le cadre du cours Méthode de Conception de Programmes, il nous a été demandé de construire un algorithme \textit{correct} qui prend en entrée un \textbf{cadran $C_{1}$} et un \textbf{entier naturel $n$} et qui génère le cadran résultant de la rotation de longueur $n$ de $C_{1}$ . \newline

L'algorithme doit répondre à certaines contraintes qui sont:
\begin{itemize}
\item il doit être le plus général possible;
\item il doit être le plus simple possible;
\item il ne doit pas utiliser d'Api Java;
\item la mémoire utilisée doit être fixe;
\item il doit avoir une complexité algorithmique temporelle en $O(n)$ où $n$ représente la taille du cadran. \newline
\end{itemize}

Ce présent rapport a donc pour but de vous présenter notre solution au problème et de vous prouvez que notre algorithme est \textbf{correct} et qu'il répond aux \textbf{contraintes} citées plus haut. Pour ce faire, la méthode utilisée est celle vue en cours qu'on peut découper en plusieurs parties:
%Expliquer différentes parties ?
\begin{itemize}
\item Théorie du problème
\item Convention de représentation
\item Spécifications des méthodes
\item Code Java
\item Correction totale de l'algorithme: triplet de Hoare-Manna et variant \newline
\end{itemize}

\section{Théorie du problème}

Soit $C$ un cadran de longueur $y$ (disposant de y éléments). Nous voulons effectuer une rotation de  longueur $n$ sur ce cadran. Une première propriété utile à la résolution de ce problème est la théorie des \textbf{modulos}. \newline

La théorie des modulos nous permet de dire, si on compte les éléments du cadran à partir de l'indice 0, de dire que l'élément 0 est le même que le yème élément pour le cadran C. L'élément 1 est égal au (y+1)éme élément du cadran C, etc. On peut le traduire sous cette écriture:\newline
i mod y = y+i mod y ou de façon plus générale \newline
i mod y = (k*y)+i mod y (où i est compris entre [0, y[ et k appartient aux entiers naturels). \newline
%Introduire schéma pour montrer cela

La deuxième théorie est la suivante:\newline
Soit $p$ le \textbf{plus grand commun diviseur} de $y$ et $n$. Nommons $a_{0}$, $a_{1}$, ..., $a_{y-1}$ les éléments du cadran. Considérons un \textbf{parcours} du cadran par pas de $n$ en démarrant de l'élément $a_{i}$ ($i \in [0,y[$) comme étant la suite d'éléments $a_{i}$ , $a_{i+n}$,$a_{i+2n}$, ... du cadran jusqu'à retomber sur l'élément de départ $a_{i}$. \newline

Nous pouvons donc énoncer la théorie du problème comme étant que les \textbf{parcours successifs} par pas de $n$ avec point de départ les éléments  $a_{i}$, ..., $a_{i+(p-1)}$ permet de visiter tout les éléments du cadran une fois sauf les éléments de départ qui seront visités au début et à la fin d'un parcours.\newline

La démonstration de cette propriété se fera en 2 parties:
\begin{itemize}
\item Montrer que chaque parcours contiendra un même \textbf{nombre d'éléments} $k+1$, où $k = y/p$.
\item Montrer que les parcours avec point de départ $a_{i}$,...,$a_{i+(p-1)}$ ne possèdent \textbf{aucun élément en commun}.\newline
\end{itemize}

Soit $k = y/p$. Pour un parcours $P$ quelconque ayant comme point de départ $a_{i}$ et pour pas $n$, il faut avoir effectuer $k$ nombre de pas de n avant de retomber sur $a_{i}$. \newline

En effet, si nous visitons les éléments par pas de $n$, il faut que la somme des pas soit égale à un multiple de $y$ pour retomber sur un même élément (on peut démontrer cela gràce aux modulos, pour que 2 nombres modulo n soit égaux il faut que leur différence soit un multiple de n). Or $n*k$ est bien un multiple de y puisque $k = y/p$ et que $p$ divise $n$. On peut résumer cela par la formule:\newline
$(n*k+i)$ mod $y = i$ \newline

Il reste à prouver que $n*k$ est bien le plus petit multiple de $y$ divisible par $n$. On a que:\newline
$nk = \frac{n*y}{p} = q*y$ (où $n=q*p$).\newline
Il n'existe pas de multiple de $y$ divisible par $n$ plus petit que $n*k$ car sinon $p$ ne serait pas le plus grand commun diviseur de $y$ et $n$.\newline

On peut conclure que chaque parcours contient alors $k+1$ éléments. \newline

Nous allons maintenant prouver que chaque éléments sont différents pour les parcours commençant à l'indice $a_{i}$ jusqu'aux parcours commençant à l'indice $a_{i+(p-1)}$.\newline

Considérons deux parcours $P_{1}$ et $P_{2}$ ayant respectivement comme point de départ $a_{i}$ et $a_{j}$ avec $0 \leq i < j \leq p-1$. Pour que les parcours $P_{1}$ et $P_{2}$ contiennent un élément en commun, il faudrait qu'il existe un élément dans le cadran tel que:\newline
$(n*t_{1})+i$ mod $y = (n*t_{2})+j$ mod $y$  (où $t_{1}$ et $t_{2}$ sont des entiers naturels)\newline
$\Leftrightarrow (n(t_{1} - t_{2})$ mod $y = j-i$ mod $y$ \newline

Si on divise les 2 membres par $p$, on obtient alors:\newline
$q(t_{1} - t_{2}) = \frac{j-i}{p}$  (où $n=q*p$).\newline

On a que $(j-i) < p$. De plus comme $i < j$, on peut supposer que $t_{1} > t_{2}$. Comme $(t_{1}-t_{2}) \geq 1$ et que $(\frac{j-i}{p}) < 1$ alors on peut conclure que cette égalité n'a aucune solution et que donc il n'y a pas d'éléments en commun pour les parcours cités plus haut.\newline

On vient de démontrer que la propriété énoncée est vraie. Comme chaque parcours contient $k = y/p$ éléments différents et qu'on effectue $p$ parcours de cette sorte en changeant d'indice de départ, on a bien qu'on visite $k*p = (y*p)/p = y $ éléments différents. Donc on aura visité tout les éléments du cadran. \newline

A cela s'ajoute une autre propriété qui découle de la dernière qui est qu'on peut décaler tout les éléments du cadran après y sauts (où un saut est un pas de n). En effet, cela revient, pour un parcours, à remplacer chaque élément dans le parcours par son précédent.
%Ajouter schéma pour illustrer cela

\section{Convention de représentation}

\section{Spécifications des méthodes}

\section{Code Java}

\lstset{language=Java}

\begin{lstlisting}[frame=single]

public static void rotation(int[] cadran, int rotation){
	int n = cadran.length;  
	//nombre de decalements restant a effectuer
	int depart = 0;		
	//indice de depart pour premier parcours
		
	while(n!=0){
		n = parcours(cadran, depart, rotation, n);
		depart++; // on change d'indice de depart
	}
}

public static int parcours(int[] cadran, 
			int depart, int pas, int n){
	int i = depart + pas;	
	int temp = 0;	
	int deplacer = cadran[i%cadran.length]; 
		
	cadran[i%cadran.length] = 
			cadran[depart%cadran.length];
	n--;
		
	while((i % cadran.length) != depart){
		temp = cadran[(i+pas)%cadran.length];
		cadran[(i+pas)%cadran.length] = 
					deplacer;
		deplacer = temp;
		i = i + pas;
		n--;
	}
	return n;
}

\end{lstlisting} 



\end{document}